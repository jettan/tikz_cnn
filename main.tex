\documentclass[a4paper, 10pt]{article}

\usepackage{graphicx}
\usepackage{color}
\usepackage{tikz}
\usepackage{pgfplots}
\usepackage{pgf-umlsd}
\usepackage{ifthen}
\usepackage[]{fp}

\FPset{totalOffset}{0}

\begin{document}

\begin{figure}
	\noindent\resizebox{\textwidth}{!}{
	\begin{tikzpicture}
		\draw[use as bounding box, transparent] (-1.8,-1.8) rectangle (17.2, 3.2);

		% Define the macro.
		% 1st argument: Height and width of the layer rectangle slice.
		% 2nd argument: Depth of the layer slice
		% 3rd argument: X Offset --> use it to offset layers from previously drawn layers.
		% 4th argument: Options for filldraw.
		% 5th argument: Text to be placed below this layer.
		% 6th argument: Y Offset --> Use it when an output needs to be fed to multiple layers that are on the same X offset.

		\newcommand{\networkLayer}[7]{
			\xdef\totalOffset{\totalOffset}
 			\ifthenelse{\equal{#7} {start}}
 			{\FPset{totalOffset}{0}}
 			{}
 			\FPeval\currentOffset{0+(totalOffset)+#3}

			\def\a{#1} % Used to distinguish input resolution for current layer.
			\def\b{0.02}
			\def\c{#2} % Width of the cube to distinguish number of input channels for current layer.
			\def\t{\currentOffset} % X offset for current layer.
			\def\d{#4} % Y offset for current layer.

			% Draw the layer body.
			\draw[line width=0.3mm](\c+\t,0,\d) -- (\c+\t,\a,\d) -- (\t,\a,\d);                                                      % back plane
			\draw[line width=0.3mm](\t,0,\a+\d) -- (\c+\t,0,\a+\d) node[midway,below] {#6} -- (\c+\t,\a,\a+\d) -- (\t,\a,\a+\d) -- (\t,0,\a+\d); % front plane
			\draw[line width=0.3mm](\c+\t,0,\d) -- (\c+\t,0,\a+\d);
			\draw[line width=0.3mm](\c+\t,\a,\d) -- (\c+\t,\a,\a+\d);
			\draw[line width=0.3mm](\t,\a,\d) -- (\t,\a,\a+\d);

			% Recolor visible surfaces
			\filldraw[#5] (\t+\b,\b,\a+\d) -- (\c+\t-\b,\b,\a+\d) -- (\c+\t-\b,\a-\b,\a+\d) -- (\t+\b,\a-\b,\a+\d) -- (\t+\b,\b,\a+\d); % front plane
			\filldraw[#5] (\t+\b,\a,\a-\b+\d) -- (\c+\t-\b,\a,\a-\b+\d) -- (\c+\t-\b,\a,\b+\d) -- (\t+\b,\a,\b+\d);

			% Colored slice.
			\ifthenelse {\equal{#5} {}}
			{} % Do not draw colored slice if #4 is blank.
			{\filldraw[#5] (\c+\t,\b,\a-\b+\d) -- (\c+\t,\b,\b+\d) -- (\c+\t,\a-\b,\b+\d) -- (\c+\t,\a-\b,\a-\b+\d);} % Else, draw a colored slice.

			\FPeval\totalOffset{0+(currentOffset)+\c}
		}

		% INPUT
		\node[] (input image) at (-3.75,0.5) {\includegraphics[height=30mm]{lenna.png}};
		\networkLayer{3.0}{0.03}{0.0}{0.0}{color=gray!80}{}{start}

		% ENCODER
		\networkLayer{3.0}{0.1}{0.5}{0.0}{color=white}{conv}{}    % S1
		\networkLayer{3.0}{0.1}{0.1}{0.0}{color=white}{}{}        % S2
		\networkLayer{2.5}{0.2}{0.1}{0.0}{color=white}{conv}{}    % S1
		\networkLayer{2.5}{0.2}{0.1}{0.0}{color=white}{}{}        % S2
		\networkLayer{2.0}{0.4}{0.1}{0.0}{color=white}{conv}{}    % S1
		\networkLayer{2.0}{0.4}{0.1}{0.0}{color=white}{}{}        % S2
		\networkLayer{1.5}{0.8}{0.1}{0.0}{color=white}{conv}{}    % S1
		\networkLayer{1.5}{0.8}{0.1}{0.0}{color=white}{}{}        % S2
		\networkLayer{1.0}{1.5}{0.1}{0.0}{color=white}{conv}{}    % S1
		\networkLayer{1.0}{1.5}{0.1}{0.0}{color=white}{}{}        % S2

		% DECODER
		\networkLayer{1.0}{1.5}{0.3}{0.0}{color=white}{deconv}{} % S1
		\networkLayer{1.0}{1.5}{0.1}{0.0}{color=white}{}{}       % S2
		\networkLayer{1.5}{0.8}{0.1}{0.0}{color=white}{deconv}{} % S1
		\networkLayer{1.5}{0.8}{0.1}{0.0}{color=white}{}{}       % S2
		\networkLayer{2.0}{0.4}{0.1}{0.0}{color=white}{}{}       % S1
		\networkLayer{2.0}{0.4}{0.1}{0.0}{color=white}{}{}       % S2
		\networkLayer{2.5}{0.2}{0.1}{0.0}{color=white}{}{}       % S1
		\networkLayer{2.5}{0.2}{0.1}{0.0}{color=white}{}{}       % S2
		\networkLayer{3.0}{0.1}{0.1}{0.0}{color=white}{}{}       % S1
		\networkLayer{3.0}{0.1}{0.1}{0.0}{color=white}{}{}       % S2

		% OUTPUT
		\networkLayer{3.0}{0.05}{0.9}{0.0}{color=red!40}{}{}          % Pixelwise segmentation with classes.
		\node[] (output image) at (18,0.5) {\includegraphics[height=30mm]{vermeer.jpg}};


	\end{tikzpicture}
	}
	\caption{Example CNN.}
	\label{fig:cnn}
\end{figure}

\end{document}
